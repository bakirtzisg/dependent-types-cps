\usepackage[utf8]{inputenc}

% margins
\usepackage[letterpaper,left=1.5in,right=1.5in,top=1in,bottom=1in]{geometry}
\RequirePackage[l2tabu, orthodox]{nag}

\usepackage{setspace}
\setstretch{1.2}

\RequirePackage[T1]{fontenc}
\usepackage[ttscale=.875]{libertine}
\usepackage[scaled=0.96]{zi4}

\usepackage{sansmath}
\usepackage{amsmath, amssymb, amsthm, mathtools}

% smaller headings
\usepackage[small]{titlesec}

% biber
\usepackage[numbers]{natbib}

% figure caption
\usepackage[labelsep=period]{caption}
\renewcommand{\captionfont}{\small\sffamily}
\renewcommand{\captionlabelfont}{\small\sffamily\bfseries}

% references
\usepackage{cleveref}

\usepackage{xcolor}
\definecolor{text_gray}{RGB}{97, 97, 97}

% math abbreviations

\newcommand{\ca}{\mathbf}
\newcommand{\inp}{\mathrm{in}}
\newcommand{\out}{\mathrm{out}}

% tikz stuff

\usepackage{tikz}
\usetikzlibrary{positioning,decorations.pathreplacing,decorations.pathmorphing,shapes.arrows,decorations.markings,arrows.meta,calc,fit,quotes,cd,math,decorations.markings,arrows,backgrounds,shapes.geometric,shapes}

\newcommand{\twocell}[3][]{\arrow[draw=none,to path={(dom#2.center)--(cod#2.center)\tikztonodes}]{}[anchor=center,#1]{\Downarrow #3}}
\newcommand{\twocellalt}[3][]{\arrow[draw=none,to path={(dom#2.center)--(cod#2.center)\tikztonodes}]{}[anchor=center,#1]{#3}}
\newcommand{\twocellA}[2][]{\twocell[#1]{A}{#2}}
\newcommand{\twocellB}[2][]{\twocell[#1]{B}{#2}}
\newcommand{\twocellC}[2][]{\twocell[#1]{C}{#2}}
\newcommand{\twocellD}[2][]{\twocell[#1]{D}{#2}}
\newcommand{\twocellE}[2][]{\twocell[#1]{E}{#2}}
\newcommand{\twocellF}[2][]{\twocell[#1]{F}{#2}}

\newcommand{\SmallBox}[3]
{\begin{tikzpicture}[oriented WD, baseline=-2.5pt, bb Small]
\node[inner sep=.1cm] [bb={1}{1}] (X) {$\scriptstyle #3$};
\draw[label] node[left=.1 of X_in1] (Y) {$#1$}
             node[right=.1 of X_out1] {$#2$};
\end{tikzpicture}}

\newcommand{\SmallBoxTwo}[4]
{\begin{tikzpicture}[oriented WD, baseline=-2.5pt, bb Small]
\node[inner sep=.1cm] [bb={2}{1}] (X) {$\scriptstyle #4$};
\draw[label] node[left=.1 of X_in1] (Y) {$#1$}
node[left=.1 of X_in2] (Y) {$#2$}
node[right=.1 of X_out1] {$#3$};
\end{tikzpicture}}

\newcommand{\SmallBoxThree}[5]
{\begin{tikzpicture}[oriented WD, baseline=-2.5pt, bb Small]
\node[inner sep=.1cm] [bb={3}{1}] (X) {$\scriptstyle #5$};
\draw[label] node[left=.1 of X_in1] (Y) {$#1$}
node[left=.1 of X_in2] (Y) {$#2$}
node[left=.1 of X_in3] (Y) {$#3$}
node[right=.1 of X_out1] {$#4$};
\end{tikzpicture}}

\newcommand*{\cocolon}{% colon with spacing for backwards arrows
  \nobreak
  \mskip6mu plus1mu
  \mathpunct{}%
  \nonscript
  \mkern-\thinmuskip
  {:}%
  \mskip2mu
  \relax
}

\tikzcdset{%changes the tip of all arrows -> in diagrams to a smaller, prettier one
   arrow style=tikz,
   diagrams={>={Classical TikZ Rightarrow[angle=63:4pt, line width=.6pt]}},
   arrows={semithick}
}

\tikzset{
  tick/.style={postaction={
    decorate,
    decoration={markings, mark=at position 0.5 with {\draw[-] (0,.4ex) -- (0,-.4ex);}}}
  },
  tickx/.style={
    postaction={ decorate,
      decoration={markings,
        mark=at position 0.5 with {
          \fill circle [radius=.28ex];
%         \draw[-] (-.35ex,-.35ex) -- (.35ex,.35ex);
%         \draw[-] (-.35ex,.35ex) -- (.35ex,-.35ex);
        }
      }
    }
  }
}
\newcommand{\tickar}
{\begin{tikzcd}[baseline=-0.5ex,cramped,sep=small,ampersand replacement=\&]{}\ar[r,tick]\&{}\end{tikzcd}}
\tikzset{
   dom/.style={append after command={coordinate[alias=dom#1]}},
   domA/.style={dom=A}, domB/.style={dom=B},
   domC/.style={dom=C}, domD/.style={dom=D},
   domE/.style={dom=E}, domF/.style={dom=F},
   cod/.style={append after command={coordinate[alias=cod#1]}},
   codA/.style={cod=A}, codB/.style={cod=B},
   codC/.style={cod=C}, codD/.style={cod=D},
   codE/.style={cod=E}, codF/.style={cod=F}
}

\tikzset{
   oriented WD/.style={%everything after equals replaces "oriented WD" in key.
      every to/.style={out=0,in=180,draw},
      label/.style={
         font=\everymath\expandafter{\the\everymath\scriptstyle},
         inner sep=0pt,
         node distance=2pt and -2pt},
      semithick,
      node distance=1 and 1,
      decoration={markings, mark=at position .5 with {\arrow{stealth};}},
      ar/.style={postaction={decorate}},
      execute at begin picture={\tikzset{
         x=\bbx, y=\bby,
         every fit/.style={inner xsep=\bbx, inner ysep=\bby}}}
      },
   bbx/.store in=\bbx,
   bbx = 1.5cm,
   bby/.store in=\bby,
   bby = 1.75ex,
   bb port sep/.store in=\bbportsep,
   bb port sep=2,
   % bb wire sep/.store in=\bbwiresep,
   % bb wire sep=1.75ex,
   bb port length/.store in=\bbportlen,
   bb port length=4pt,
   bb min width/.store in=\bbminwidth,
   bb min width=1cm,
   bb rounded corners/.store in=\bbcorners,
   bb rounded corners=2pt,
   bb small/.style={bb port sep=1, bb port length=2.5pt, bbx=.4cm, bb min width=.4cm, bby=.7ex},
   bb Small/.style={bb port sep=1, bb port length=2.5pt, bbx=.5cm, bb min width=.5cm, bby=1ex},
   bb/.code 2 args={%When you see this key, run the code below:
      \pgfmathsetlengthmacro{\bbheight}{\bbportsep * (max(#1,#2)+1) * \bby}
      \pgfkeysalso{draw,minimum height=\bbheight,minimum width=\bbminwidth,outer sep=0pt,
         rounded corners=\bbcorners,thick,
         prefix after command={\pgfextra{\let\fixname\tikzlastnode}},
         append after command={\pgfextra{\draw
            \ifnum #1=0{} \else foreach \i in {1,...,#1} {
               ($(\fixname.north west)!{\i/(#1+1)}!(\fixname.south west)$) +(-\bbportlen,0) coordinate
               (\fixname_in\i) -- +(\bbportlen,0) coordinate (\fixname_in\i')}\fi %Define the endpoints of tickmarks
            \ifnum #2=0{} \else foreach \i in {1,...,#2} {
               ($(\fixname.north east)!{\i/(#2+1)}!(\fixname.south east)$) +(-\bbportlen,0) coordinate
               (\fixname_out\i') -- +(\bbportlen,0) coordinate (\fixname_out\i)}\fi;
         }}}
   },
   bb name/.style={append after command={\pgfextra{\node[anchor=north] at (\fixname.north) {#1};}}}
}

\newcommand{\SmallBoxTwoOut}[4]
{\begin{tikzpicture}[oriented WD, baseline=(Y.south), bb Small]
\node[inner sep=.1cm] [bb={1}{2}] (X) {$\scriptstyle #4$};
\draw[label] node[left=.1 of X_in1] {$#1$}
node[right=.1 of X_out1] {$#2$}
node[right=.1 of X_out2] {$#3$};
\end{tikzpicture}}
