\documentclass{article}

\usepackage[utf8]{inputenc}

% margins
\usepackage[letterpaper,left=1.5in,right=1.5in,top=1in,bottom=1in]{geometry}
\RequirePackage[l2tabu, orthodox]{nag}

\usepackage{setspace}
\setstretch{1.2}

\RequirePackage[T1]{fontenc}
\usepackage[ttscale=.875]{libertine}
\usepackage[scaled=0.96]{zi4}

\usepackage{sansmath}
\usepackage{amsmath, amssymb, amsthm, mathtools}

% smaller headings
\usepackage[small]{titlesec}

% biber
\usepackage[numbers]{natbib}

% figure caption
\usepackage[labelsep=period]{caption}
\renewcommand{\captionfont}{\small\sffamily}
\renewcommand{\captionlabelfont}{\small\sffamily\bfseries}

% references
\usepackage{cleveref}


\frenchspacing
\begin{document}

Categorical systems theory is a mathematical framework for organizing the
composition and coupling of many different sorts of systems which arise in
scientific and engineering practice. This formal framework for composition
allows us to prove \emph{compositionality theorems} which formulate
behaviors of composite systems in terms of the behaviors of their component systems.

There are many various sorts of systems in use --- differential equations, Moore
machines, Markov decision processes, labelled transition systems, flow charts,
port Hamiltonian systems, and more --- and all can be organized into a
\emph{doubly indexed category}.\footnote{I'm actually going to describe
  something slightly different than what I do in my book, which I might more
  accurately describe as an \emph{algebra for a double operad} --- but it's
  basically the same thing as a monoidal doubly indexed category.} A doubly indexed category of systems consists
of:
\begin{itemize}
  \item A monoidal category of interfaces and composition patters. An
    interface is an abstract model of the way in which a system can interface
    with its environment. In most examples, it consists of a finite set of
    ports, perhaps labeled with the type of data which may pass through the
    port. These ports may also be separated into inputs and outputs.

    A composition pattern is a way that a finite set of interfaces can be wired
    together inside another interfact; that is, a pattern by which systems with
    those interfaces can be composed by interacting through their interfaces. We
    will discuss both of these notions and examples in Section
    \ref{sec.interfaces}.
    \item To every interface $I$, a set $\textbf{Sys}(I)$ of systems with that interface. These
      systems should all be of a particular sort --- differential equations,
      Moore machines, etc. --- but we will discuss ways that different sorts of
      systems can be brought together using \emph{doubly indexed functors} in
      order to design systems with components of multiple different sorts.
    \item To every composition pattern $c : I_1,\ldots, I_n \to J$, a function
      \[
        c_{\ast} : \textbf{Sys}(I_1) \times \cdots \times \textbf{Sys}(I_n) \to \textbf{Sys}(J)
      \]
      which sends systems $S_i$ with interface $I_i$ to their composite
      $c_{\ast}(S_1, \ldots, S_n)$ which has interface $J$. [Graphically, this
      can be thought of as filling in the wiring diagram boxes with the systems $S_i$.].
\end{itemize}
So far, the sort of structure described has been described for this purpose
before as a $W$-algebra for an operad $W$ of wiring diagrams (another name for
composition patterns). But what the notion of doubly indexed category adds is
the \emph{maps} between systems, which include general simulations of systems by
others, trajectories, steady states, periodic orbits, and more. Importantly, a
map of systems can change the interface; for this reason, we need a notion of
map of interfaces.
\begin{itemize}
  \item In addition to composition patterns showing how interfaces $I_1,\ldots,
    I_n$ may be wired together inside $J$, we also need \emph{maps} of
    interfaces. In most cases these maps $f : I \to J$ of interfaces are functions from the
    data types of the ports on $I$ to the data types of the ports on $J$. 

    We furthermore need a notion of \emph{compatibility} between composition
    patterns and maps of interfaces which tells us when a map $f : I \to I'$ on
    composent systems induces a map $g : J \to J'$ via the composition patterns
    $c : I \to J$ and $c' : I' \to J'$. The result is a \emph{double category}
    of interfaces. We
    will discuss this further in Section
    \ref{sec.interfaces}.
  \item For every map $f : I \to I'$ between interfaces, we need a
    notion of maps $\phi : S \to S'$ from an $I$-system $S$ to an $I'$-system
    $S'$ which acts exactly as $f$ on their respective interfaces. This gives us
    a matrix of sets (or span) $\textbf{Sys}(f) :
    \textbf{Sys}(I) \times \textbf{Sys}(I') \to \textbf{Set}$ which
    assigns to an $I$-system $S$ and $I'$-system $S'$ the set
    $\textbf{Sys}(f)(S, S')$ of maps from $S$ to $S'$. 
\end{itemize}

\section{Double Categories of Interfaces and Composition Patterns} \label{sec.interfaces}



\section{Doubly Indexed Categories of Systems}


\section{Behaviors of Systems via Representable Functors}

\bibliography{manuscript.bib}
\bibliographystyle{plain}

\end{document}